\documentclass[oneside,final,14pt]{extreport}
\usepackage[utf8x]{inputenc}
\usepackage[T2A]{fontenc}
\usepackage{vmargin}
\usepackage{indentfirst}
\usepackage[russianb]{babel}
\usepackage{amsmath}
\usepackage{amsfonts}
\setpapersize{A4}
\setmarginsrb{3cm}{2cm}{1cm}{2cm}{0pt}{0mm}{0pt}{13mm}
\linespread{1.5}
\sloppy

\title{Проект по ознакомительной практике.}
\author{Ярослав}
\date{May 2022}

\begin{document}

\maketitle

\chapter{Теоретическая часть}

	\section{Основы криптографии}

	Слово \textbf{криптография} происходит от греческих слов, означающих "<скрытное письмо">.
	У криптографии долгая и красочная история, насчитывающая несколько тысяч лет.
	
	С профессиональной точки зрения понятия "<шифр"> и "<код"> отличаются друг от друга. \textbf{Шифр} (cipher) представляет собой посимвольное или побитовое преобразование, не зависящее от лингвистической структуры сообщения. \textbf{Код} (code), напротив, заменяет целое слово другим словом или символом. 
	Коды в настоящее время не используются.
	Сейчас в криптографии нуждается каждая цифровая область. 
	Ни один пакет в сети не посылается в открытом виде (в хорошем случае), ни один диск, имеющий важную информацию, не хранит данные незашфированными.

	До появления компьютеров одним из основных сдерживающих факторов в криптографии была возможность шифровальщика выполнить необходимые преобразования, часто на поле боя, с помощью несложного оборудования.
	Кроме того, достаточно сложной задачей было быстрое переключение с одного криптографического метода на другой, так как для этого требовалось переобучение большого количества людей.
	Тем не менее опасность того, что шифровальщик может быть захвачен противником, заставила постоянно развивать способы смены криптографических методов при необходимости.
	Эти противоречивые требования приводят к модели процесса шифрования—дешифрации.
	Незашифрованный текст называют \textbf{Открытым} (plain text), а зашифрованный~--- \textbf{зашифрованным текстом} (chiper text). Есть ещё такое понятие, как \textbf{ключ} (key).

	Основное правило криптографии состоит в предположении, что криптоаналитику (взломщику шифра) известен используемый метод шифрования. Другими словами, злоумышленник точно знает, как работают методы шифрования и дешифрации.

	\large{Принципы Керкгофа:}
	\begin{enumerate}
		\item 
	\end{enumerate}





	\section{Популярные алгоритмы фиксации контрольных сумм}

		\subsection{MD5}

		\subsection{SHA1/2}

	\section{Отечественная криптография}

		\subsection{Магама}

		\subsection{Кузнечик}

		\subsection{Стриборг}

\chapter{Практическая часть}


\end{document}

